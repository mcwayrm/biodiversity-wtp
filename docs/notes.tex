% Options for packages loaded elsewhere
\PassOptionsToPackage{unicode}{hyperref}
\PassOptionsToPackage{hyphens}{url}
\PassOptionsToPackage{dvipsnames,svgnames,x11names}{xcolor}
%
\documentclass[
  letterpaper,
]{article}

\usepackage{amsmath,amssymb}
\usepackage{iftex}
\ifPDFTeX
  \usepackage[T1]{fontenc}
  \usepackage[utf8]{inputenc}
  \usepackage{textcomp} % provide euro and other symbols
\else % if luatex or xetex
  \usepackage{unicode-math}
  \defaultfontfeatures{Scale=MatchLowercase}
  \defaultfontfeatures[\rmfamily]{Ligatures=TeX,Scale=1}
\fi
\usepackage{lmodern}
\ifPDFTeX\else  
    % xetex/luatex font selection
\fi
% Use upquote if available, for straight quotes in verbatim environments
\IfFileExists{upquote.sty}{\usepackage{upquote}}{}
\IfFileExists{microtype.sty}{% use microtype if available
  \usepackage[]{microtype}
  \UseMicrotypeSet[protrusion]{basicmath} % disable protrusion for tt fonts
}{}
\makeatletter
\@ifundefined{KOMAClassName}{% if non-KOMA class
  \IfFileExists{parskip.sty}{%
    \usepackage{parskip}
  }{% else
    \setlength{\parindent}{0pt}
    \setlength{\parskip}{6pt plus 2pt minus 1pt}}
}{% if KOMA class
  \KOMAoptions{parskip=half}}
\makeatother
\usepackage{xcolor}
\setlength{\emergencystretch}{3em} % prevent overfull lines
\setcounter{secnumdepth}{-\maxdimen} % remove section numbering
% Make \paragraph and \subparagraph free-standing
\makeatletter
\ifx\paragraph\undefined\else
  \let\oldparagraph\paragraph
  \renewcommand{\paragraph}{
    \@ifstar
      \xxxParagraphStar
      \xxxParagraphNoStar
  }
  \newcommand{\xxxParagraphStar}[1]{\oldparagraph*{#1}\mbox{}}
  \newcommand{\xxxParagraphNoStar}[1]{\oldparagraph{#1}\mbox{}}
\fi
\ifx\subparagraph\undefined\else
  \let\oldsubparagraph\subparagraph
  \renewcommand{\subparagraph}{
    \@ifstar
      \xxxSubParagraphStar
      \xxxSubParagraphNoStar
  }
  \newcommand{\xxxSubParagraphStar}[1]{\oldsubparagraph*{#1}\mbox{}}
  \newcommand{\xxxSubParagraphNoStar}[1]{\oldsubparagraph{#1}\mbox{}}
\fi
\makeatother


\providecommand{\tightlist}{%
  \setlength{\itemsep}{0pt}\setlength{\parskip}{0pt}}\usepackage{longtable,booktabs,array}
\usepackage{calc} % for calculating minipage widths
% Correct order of tables after \paragraph or \subparagraph
\usepackage{etoolbox}
\makeatletter
\patchcmd\longtable{\par}{\if@noskipsec\mbox{}\fi\par}{}{}
\makeatother
% Allow footnotes in longtable head/foot
\IfFileExists{footnotehyper.sty}{\usepackage{footnotehyper}}{\usepackage{footnote}}
\makesavenoteenv{longtable}
\usepackage{graphicx}
\makeatletter
\newsavebox\pandoc@box
\newcommand*\pandocbounded[1]{% scales image to fit in text height/width
  \sbox\pandoc@box{#1}%
  \Gscale@div\@tempa{\textheight}{\dimexpr\ht\pandoc@box+\dp\pandoc@box\relax}%
  \Gscale@div\@tempb{\linewidth}{\wd\pandoc@box}%
  \ifdim\@tempb\p@<\@tempa\p@\let\@tempa\@tempb\fi% select the smaller of both
  \ifdim\@tempa\p@<\p@\scalebox{\@tempa}{\usebox\pandoc@box}%
  \else\usebox{\pandoc@box}%
  \fi%
}
% Set default figure placement to htbp
\def\fps@figure{htbp}
\makeatother

\makeatletter
\@ifpackageloaded{tcolorbox}{}{\usepackage[skins,breakable]{tcolorbox}}
\@ifpackageloaded{fontawesome5}{}{\usepackage{fontawesome5}}
\definecolor{quarto-callout-color}{HTML}{909090}
\definecolor{quarto-callout-note-color}{HTML}{0758E5}
\definecolor{quarto-callout-important-color}{HTML}{CC1914}
\definecolor{quarto-callout-warning-color}{HTML}{EB9113}
\definecolor{quarto-callout-tip-color}{HTML}{00A047}
\definecolor{quarto-callout-caution-color}{HTML}{FC5300}
\definecolor{quarto-callout-color-frame}{HTML}{acacac}
\definecolor{quarto-callout-note-color-frame}{HTML}{4582ec}
\definecolor{quarto-callout-important-color-frame}{HTML}{d9534f}
\definecolor{quarto-callout-warning-color-frame}{HTML}{f0ad4e}
\definecolor{quarto-callout-tip-color-frame}{HTML}{02b875}
\definecolor{quarto-callout-caution-color-frame}{HTML}{fd7e14}
\makeatother
\makeatletter
\@ifpackageloaded{caption}{}{\usepackage{caption}}
\AtBeginDocument{%
\ifdefined\contentsname
  \renewcommand*\contentsname{Table of contents}
\else
  \newcommand\contentsname{Table of contents}
\fi
\ifdefined\listfigurename
  \renewcommand*\listfigurename{List of Figures}
\else
  \newcommand\listfigurename{List of Figures}
\fi
\ifdefined\listtablename
  \renewcommand*\listtablename{List of Tables}
\else
  \newcommand\listtablename{List of Tables}
\fi
\ifdefined\figurename
  \renewcommand*\figurename{Figure}
\else
  \newcommand\figurename{Figure}
\fi
\ifdefined\tablename
  \renewcommand*\tablename{Table}
\else
  \newcommand\tablename{Table}
\fi
}
\@ifpackageloaded{float}{}{\usepackage{float}}
\floatstyle{ruled}
\@ifundefined{c@chapter}{\newfloat{codelisting}{h}{lop}}{\newfloat{codelisting}{h}{lop}[chapter]}
\floatname{codelisting}{Listing}
\newcommand*\listoflistings{\listof{codelisting}{List of Listings}}
\makeatother
\makeatletter
\makeatother
\makeatletter
\@ifpackageloaded{caption}{}{\usepackage{caption}}
\@ifpackageloaded{subcaption}{}{\usepackage{subcaption}}
\makeatother

\usepackage{bookmark}

\IfFileExists{xurl.sty}{\usepackage{xurl}}{} % add URL line breaks if available
\urlstyle{same} % disable monospaced font for URLs
\hypersetup{
  pdftitle={The Economic Value of Biodiversity in India},
  pdfauthor={Raahil Madhok; Matt Braaksma; Ryan McWay; Jovin Lasway},
  colorlinks=true,
  linkcolor={blue},
  filecolor={Maroon},
  citecolor={Blue},
  urlcolor={Blue},
  pdfcreator={LaTeX via pandoc}}


\title{The Economic Value of Biodiversity in India}
\usepackage{etoolbox}
\makeatletter
\providecommand{\subtitle}[1]{% add subtitle to \maketitle
  \apptocmd{\@title}{\par {\large #1 \par}}{}{}
}
\makeatother
\subtitle{Running Notes}
\author{Raahil Madhok \and Matt Braaksma \and Ryan McWay \and Jovin
Lasway}
\date{}

\begin{document}
\maketitle

\renewcommand*\contentsname{Table of contents}
{
\hypersetup{linkcolor=}
\setcounter{tocdepth}{3}
\tableofcontents
}

\subsection{Project Description}\label{project-description}

This project aims to produce the first measure of willingness to pay
(WTP) for biodiversity. This provides a monetary measure to value
biodiversity which can be applied to a wide variety of applications in
environmental economics and natural captial accounting. The data relies
on E-bird observations of diverse bird species observations. The sample
is limited to local, long-time users who are residents (not tourists) in
India. The methodology to calculate WTP relies on a revealed preference
throughh the random utility model (RUM). The RUM determines the value of
seeking out diverse species of birds by comparing the value of
alternative counterfactual bird siting locations and the cost to go to
these counterfactuals. Monetary value is determined through the RUM by
estimating the travel cost of the reveal preference compared to
counterfactual locations.

\subsection{Notes}\label{notes}

For notes, we present the most recent first so that notes are
chronologically most recent.

\begin{tcolorbox}[enhanced jigsaw, bottomrule=.15mm, coltitle=black, opacityback=0, opacitybacktitle=0.6, titlerule=0mm, leftrule=.75mm, arc=.35mm, toptitle=1mm, title=\textcolor{quarto-callout-note-color}{\faInfo}\hspace{0.5em}{8/15/2025}, bottomtitle=1mm, colbacktitle=quarto-callout-note-color!10!white, breakable, colback=white, colframe=quarto-callout-note-color-frame, rightrule=.15mm, toprule=.15mm, left=2mm]

\begin{quote}
Ryan, Matt, Raahil
\end{quote}

\section{TODO:}\label{todo}

\begin{itemize}
\tightlist
\item
  \textbf{Matt}: Demean data for State x month fixed effect and user
  fixed effects
\item
  \textbf{Matt}: Resolve imputing species richness by estimating
  correlation of e-bird to bird life data and deflate value

  \begin{itemize}
  \tightlist
  \item
    \textbf{Matt}: Degree of Imputation: Summary stats on \% of missing
    in observed and counterfactual data
  \end{itemize}
\item
  \textbf{Jovin}: Constructing the shift-share IV
\end{itemize}

\subsection{Agenda}\label{agenda}

\begin{enumerate}
\def\labelenumi{\arabic{enumi}.}
\tightlist
\item
  Estimation of WTP

  \begin{itemize}
  \tightlist
  \item
    negative values came from mismatched polygons

    \begin{itemize}
    \tightlist
    \item
      Difference from the clipped calipers instead of the original
      simple circles
    \end{itemize}
  \item
    weeks sometimes were only 53 (not 54). Solved with iso-week
    formating that allows for more merging of site attributes
  \end{itemize}
\item
  Data issue

  \begin{itemize}
  \tightlist
  \item
    we don't observe species richness
  \item
    So we are imputing with range maps
  \item
    When there is missing, we currently fill with species map
  \item
    Possible species ranges is always higher than recorded value from
    e-birders
  \item
    So the imputed values always are super high when missing, and then
    this makes our estimates really weird.

    \begin{itemize}
    \tightlist
    \item
      This penalizes the choice. Because counterfactual choice has huge
      species richness
    \end{itemize}
  \item
    Potential Solution: replace missing with historical average (or
    minimum average) for each district or state in the same time period.

    \begin{itemize}
    \tightlist
    \item
      Impute that for hotspots with missing data
    \end{itemize}
  \item
    Potential Solution: correlation of bird life data and e-bird
    observations

    \begin{itemize}
    \tightlist
    \item
      Then we can just deflate imputed value that rescales this to the
      e-bird values
    \end{itemize}
  \item
    What is the main source of missing value?

    \begin{itemize}
    \tightlist
    \item
      want summary stats on \% of missing in observed and counterfactual
      data
    \item
      Could make this seasonal species richness for imputed value
    \end{itemize}
  \end{itemize}
\item
  Fixed Effects Issue

  \begin{itemize}
  \tightlist
  \item
    NA for standard errors when adding individual fixed effects
  \item
    Running with python on MSI
  \item
    Could attempt to manually demean data (user fixed effects) before
    running
  \item
    State x month fixed effect and user fixed effects

    \begin{itemize}
    \tightlist
    \item
      Robustness by district x month fixed effects
    \item
      This will just be more intense
    \end{itemize}
  \item
    mlogit does not allow for fixed effects
  \item
    The key is to demean within each group

    \begin{itemize}
    \tightlist
    \item
      hdfe is package in Stata
    \end{itemize}
  \end{itemize}
\item
  Instrumental Variables

  \begin{itemize}
  \tightlist
  \item
    Have a map data for bird migration routes

    \begin{itemize}
    \tightlist
    \item
      Only found 2 species public for India
    \item
      There are other species we can request access to
    \end{itemize}
  \item
    Data avaliable for avian flu outbreak

    \begin{itemize}
    \tightlist
    \item
      H5N1
    \item
      2005 - 2025
    \end{itemize}
  \item
    Combining the two data sets we can assign outbreaks to specific
    species that migrate to India
  \item
    Human cases of avian flu deaths should be controlled for in IV
    estimation
  \item
    If we have a shift-share IV, do we need to know the flyways?

    \begin{itemize}
    \tightlist
    \item
      If this is the only channel, then we don't need to actually need
      to know the origin of the bird
    \end{itemize}
  \end{itemize}
\end{enumerate}

\[
IV_{dt} = shift_t \times share_d
\]

\[
IV_{dt} = \text{Wild Avian Flu}_t \times \text{% of species pool from migration}_d
\]

\[
Shift_t = \sum_{j = country}^J flu_{jt} \times \gamma_j
\]

\[
\gamma_j = \sum_{s =\text{total \# of species in j}}^S \frac{\text{species range overlap India}_s}{S}
\]

\begin{itemize}
\tightlist
\item
  Assumes a uniform distribution of species across their species range

  \begin{itemize}
  \tightlist
  \item
    This is an improvement over distance from India as the weight
  \end{itemize}
\end{itemize}

\[
Share_d = \sum_d \frac{Indicator for migratory}{\text{count of potential species}}
\]

\end{tcolorbox}

\begin{tcolorbox}[enhanced jigsaw, bottomrule=.15mm, coltitle=black, opacityback=0, opacitybacktitle=0.6, titlerule=0mm, leftrule=.75mm, arc=.35mm, toptitle=1mm, title=\textcolor{quarto-callout-note-color}{\faInfo}\hspace{0.5em}{8/1/2025}, bottomtitle=1mm, colbacktitle=quarto-callout-note-color!10!white, breakable, colback=white, colframe=quarto-callout-note-color-frame, rightrule=.15mm, toprule=.15mm, left=2mm]

\begin{quote}
Ryan, Matt, Raahil
\end{quote}

\section{TODO:}\label{todo-1}

\begin{itemize}
\tightlist
\item
  \textbf{Matt}: Re-estimating WTP using new sample of users (expanded)
\item
  \textbf{Matt}: Adding a Visual: Attempting to run the simple
  regression of biodiversity and hotspot visitations
\item
  \textbf{Jovin}: Complete the IV design by finding accompanying data.
  Maps of migration patterns. Papers to cite for this phenomenon.
\item
  \textbf{Raahil/Ryan}: Write up the model in LaTeX
\end{itemize}

\subsection{Agenda}\label{agenda-1}

\begin{enumerate}
\def\labelenumi{\arabic{enumi}.}
\tightlist
\item
  Figures describing variation

  \begin{itemize}
  \tightlist
  \item
    Ryan presented some post estimation figures we tabled for later
  \item
    Matt presented some maps showing variation over space
  \item
    Raahil suggested we produce a margins plots. We run an endogenous
    TWFE model of species richness on number of visitors. Then we plot
    the marginal effects at different levels of biodiversity.
  \end{itemize}
\item
  First estimates of WTP using super users

  \begin{itemize}
  \tightlist
  \item
    Matt is able to run the conditional logit model. Not able to run the
    mixed logit model
  \item
    Our \(\alpha\) for price has the wrong sign. This makes biodiversity
    WTP a negative value
  \item
    We think that the sample may be too restrictive resulting in weird
    results
  \item
    Additionally, Matt is working through some data issues in the
    counterfactual data set that is dropping 1.1\% of site attributes
  \item
    And there is colinearity amongst site attributes when running the
    conditional logit, so they are currently dropped in the model run.
  \end{itemize}
\item
  Modeling updates using IV method

  \begin{itemize}
  \tightlist
  \item
    Ryan has a clear idea of how to model the IV into our mixed logit
    model
  \item
    The key is to rely on Jacob Bradt's model using control functions.
  \item
    We estimate the first stage linear regression. Residualize. And use
    the residuals in our discrete choice method (mixed logit).
  \item
    We will need to bootstrap the 2nd stage in our to recover standard
    errors.
  \end{itemize}
\item
  Bird Migration Instruments

  \begin{itemize}
  \item
    \begin{enumerate}
    \def\labelenumii{(\roman{enumii})}
    \tightlist
    \item
      Migration patterns: From the literature I've compiled so far:
    \end{enumerate}

    \begin{itemize}
    \tightlist
    \item
      Peak arrivals: August - November
    \item
      Peak departures: February - April
    \item
      Low migration periods: May - July, and December - January
    \end{itemize}
  \item
    The Inbound migration: Birds arrive in India primarily from the
    northern breeding grounds, especially in Mongolia, Kazakhstan, and
    Siberia, during the August-November window to overwinter in wetlands
    and coastal areas.
  \item
    The Outbound migration: Birds depart February-April, returning to
    their breeding sites for nesting.
  \item
    \begin{enumerate}
    \def\labelenumii{(\roman{enumii})}
    \setcounter{enumii}{1}
    \tightlist
    \item
      Drivers of bird migration and IV framework: Ecology literature
      indicates that migration along the Central Asian Flyway is mainly
      driven by (i) Ecological reasons, india's wetlands, lakes, and
      agricultural landscapes provide abundant food and shelter during
      winter, (ii) Climatic reasons: Wrmer winter conditions compared to
      the northern latitudes reduce suruval stress (iii) Stopover
      importance: Key sites such as Chilika, Keoladeo, Pong Dam, etc
      serve as staging areas for fueling before onward flights.
    \end{enumerate}
  \item
    Recent literature uses environmental shocks as instruments (Noack et
    al., 2024; Meng et al, 2025). For India's bird migration context,
    two IV strategies appear promising.

    \begin{itemize}
    \tightlist
    \item
      Weather shock IV: Baseline district exposure (from our eBird)
      interacting with NVDI and weather anomalies in the breeding
      grounds.
    \item
      Disease shock IV: I find this is more interesting because a recent
      work by Yang et al., 2024 at Nature Communications finds avian
      influenza outbreaks do not trigger migration but do affect
      survival and presence of birds at the Indian stopovers to a very
      large extent. This makes it a credible instrument because: (i)
      outbreaks are exogenous to the local economic variables and (ii)
      outbreak timing is tied to a peak of migration window/CAF
      (Aug-Nov, Feb-Apr) in India, allowing shocks to coincide with
      relevant bird presence.
    \end{itemize}
  \item
    \begin{enumerate}
    \def\labelenumii{(\roman{enumii})}
    \setcounter{enumii}{2}
    \tightlist
    \item
      Data collection: Potential outbreak data sources, shapefiles
      include (i) FAO EMPRES-i+; (ii) WOAH/WAHIS, which tracks the
      outbreak locations, date, disease type, host species, and status.
      Lastly, GISAID shapefile data, which is an avian influenza
      sequence database, tracks the outbreak locations (used in Elbe \&
      Buckland-Merrett, 2017, Global Challenges Journal; Yang et al,
      2024, Nature Communications).
    \end{enumerate}
  \item
    \begin{enumerate}
    \def\labelenumii{(\roman{enumii})}
    \setcounter{enumii}{3}
    \tightlist
    \item
      Next step: Performing a spatial join between outbreak data and our
      ebird-based hotspots, aggregating outbreak counts to the
      district/hotspot-year panel level. I think this will allow us to
      construct the avian influenza IV as = Baseline exposure in
      district/hotspot d x outbreak shock at time t.
    \end{enumerate}
  \end{itemize}
\end{enumerate}

\end{tcolorbox}

\begin{tcolorbox}[enhanced jigsaw, bottomrule=.15mm, coltitle=black, opacityback=0, opacitybacktitle=0.6, titlerule=0mm, leftrule=.75mm, arc=.35mm, toptitle=1mm, title=\textcolor{quarto-callout-note-color}{\faInfo}\hspace{0.5em}{6/2/2025}, bottomtitle=1mm, colbacktitle=quarto-callout-note-color!10!white, breakable, colback=white, colframe=quarto-callout-note-color-frame, rightrule=.15mm, toprule=.15mm, left=2mm]

\begin{quote}
Ryan, Matt, Jovin, Raahil
\end{quote}

\section{TODO:}\label{todo-2}

\begin{itemize}
\tightlist
\item
  \textbf{Matt}: Finish species richness site attribute (cluster-time
  level)
\item
  \textbf{Matt}: Finish computing congestion site attribute
\item
  \textbf{Matt}: Check why sample size so big. Check number of elements
  in the choice set for validation.
\item
  \textbf{Matt/Ryan}: Compute species richness in radius \emph{around
  home} (varying at same time unit as other species richness attribute)
\item
  \textbf{Jovin}: Fix travel cost. I suggest checking it over with
  Matt/Ryan
\item
  \textbf{Raahil/Ryan}: Work through paper framing
\end{itemize}

\subsection{Agenda}\label{agenda-2}

\begin{enumerate}
\def\labelenumi{\arabic{enumi}.}
\tightlist
\item
  Species Richness
\end{enumerate}

\begin{itemize}
\tightlist
\item
  Maps for imputation can't have the shannon index. Because we don't
  have abundance counts
\item
  But we can do an index for the e-bird data hotspots
\item
  Estimates of bird life and expected species richness
\item
  Monthly values or seasonal values
\end{itemize}

\begin{enumerate}
\def\labelenumi{\arabic{enumi}.}
\setcounter{enumi}{1}
\tightlist
\item
  Other Site Attributes
\end{enumerate}

\begin{itemize}
\tightlist
\item
  Have one site data set for temp, rain, and tree cover
\item
  This is at the hotspot cluster level
\end{itemize}

\begin{enumerate}
\def\labelenumi{\arabic{enumi}.}
\setcounter{enumi}{2}
\tightlist
\item
  Congestion
\end{enumerate}

\begin{itemize}
\tightlist
\item
  Ready to go code, just not run yet
\item
  \# of users at hotspot cluster monthly
\end{itemize}

\begin{enumerate}
\def\labelenumi{\arabic{enumi}.}
\setcounter{enumi}{3}
\tightlist
\item
  Travel Cost
\end{enumerate}

\begin{itemize}
\tightlist
\item
  Jovin has choice set and code that estmates this for a first attempt
\item
  Matched to GDP data to impute wages
\item
  Estimate Euclidean distance between households and hotspot clusters
\item
  Final data for RUM should be choice set dimensions (for our checks)
\end{itemize}

\begin{enumerate}
\def\labelenumi{\arabic{enumi}.}
\setcounter{enumi}{4}
\tightlist
\item
  Measure of Experience at the Individual Level
\end{enumerate}

\begin{itemize}
\tightlist
\item
  Counts of trips (counter)
\item
  Count for \# of different hotspots (counter)
\item
  So ratio tells us if you have preference for location (``Habit
  Formation'').

  \begin{itemize}
  \tightlist
  \item
    Low ratio is little location preference
  \item
    High ratio is high location preference
  \end{itemize}
\end{itemize}

\begin{enumerate}
\def\labelenumi{\arabic{enumi}.}
\setcounter{enumi}{5}
\tightlist
\item
  Conceptual
\end{enumerate}

\begin{itemize}
\tightlist
\item
  Birds vs.~non-bird still valued as travel cost. So both are still
  measures of nature recreation.
\item
  Monetizing value \textgreater\textgreater{} Add in contribute to GEP
  or ecosystem service value literature.
\item
  Novelty: Why is WTP lower in Developing Countries? Welfare for
  environmental quality in envirodevonomics framework (Env-Dev framing)
\end{itemize}

\end{tcolorbox}

\begin{tcolorbox}[enhanced jigsaw, bottomrule=.15mm, coltitle=black, opacityback=0, opacitybacktitle=0.6, titlerule=0mm, leftrule=.75mm, arc=.35mm, toptitle=1mm, title=\textcolor{quarto-callout-note-color}{\faInfo}\hspace{0.5em}{5/16/2025}, bottomtitle=1mm, colbacktitle=quarto-callout-note-color!10!white, breakable, colback=white, colframe=quarto-callout-note-color-frame, rightrule=.15mm, toprule=.15mm, left=2mm]

\begin{quote}
Ryan, Matt, Jovin, Raahil
\end{quote}

\section{TODO:}\label{todo-3}

\begin{itemize}
\tightlist
\item
  Seperate meetings for Matt/Jovin (coding) and Ryan/Raahil (conception)
  for our specializations
\end{itemize}

\subsection{Agenda}\label{agenda-3}

\begin{enumerate}
\def\labelenumi{\arabic{enumi}.}
\tightlist
\item
  Finalize the site attribute data and create the master dataset
\end{enumerate}

\begin{itemize}
\tightlist
\item
  Matt and Ryan -- complete by next meeting (May 16)
\item
  We have monthly temperature and rainfall
\item
  We have annual tree cover data
\item
  These are as zonal statistics (average for 10KM radius of hotspot)

  \begin{itemize}
  \tightlist
  \item
    In E-bird, we have the average distance walked around hotspot
  \item
    Or do we avoid this work because temp, rain, and tree cover does not
    vary much at such a small spatial scale.
  \item
    Temp, rainfall, and tree are at higher resolutions. So as long as
    the radius is less than the input data this will not vary.
  \end{itemize}
\item
  We need to think more about mapping of hotspots and hotspot clusters.
\item
  Need to add a fourth site attribute: congestion

  \begin{itemize}
  \tightlist
  \item
    How frequently other users are using this site?
  \item
    Aggregate number of user trips for each hotspot but will need to do
    this as a lag (week before)
  \item
    So this is expected congestion.
  \end{itemize}
\item
  We need to transform this into a panel of counterfactual sets appended
  to the observed trips.

  \begin{itemize}
  \tightlist
  \item
    Not balanced across counterfactual trips and time
  \end{itemize}
\end{itemize}

\begin{enumerate}
\def\labelenumi{\arabic{enumi}.}
\setcounter{enumi}{1}
\tightlist
\item
  Calculate travel costs -- see notes for guidance; this will be added
  to the site attribute script
\end{enumerate}

\begin{itemize}
\tightlist
\item
  For GDP per capita data, use gridded data from here and extract the
  value closest to each users' home. Then use the formula from
  Kolstoe/cameron or lloyd-smith to calculate travel costs.
\item
  Jovin -- complete by next meeting (May 16)
\item
  Collected 2014-2024 GDP per capita data
\item
  Laptop trouble estimating, working with MPC.
\item
  Could run the travel cost with a small random sample to make this
  managable for Jovin's laptop capacity.
\item
  Can re-run the pipeline so we are running for only frequent users
  (1.6M instead of the 3M)
\end{itemize}

\begin{enumerate}
\def\labelenumi{\arabic{enumi}.}
\setcounter{enumi}{2}
\tightlist
\item
  Species richness
\end{enumerate}

\begin{itemize}
\tightlist
\item
  Matt and Ryan -- bring three ideas for next meeting (May 16)
\item
  Close to being done (Matt taking the lead)
\item
  Taking raw e-bird data and getting monthly counts
\item
  Need to translate from hotspots to hotspot clusters
\item
  Species ranges (for missing) shapefiles are crashing the code.
\item
  All of this is to get a species at a hotspot so we can turn this into
  a richness index

  \begin{itemize}
  \tightlist
  \item
    Will want this merged as richness for same month/week from previous
    year
  \item
    In the app, you can see in real-time species count from their
    favorite places.
  \item
    So likely quite responsive. Last week is best
  \item
    Can also do monthly to replicate results from month last year from
    Kolsto paper method.
  \end{itemize}
\item
  We aim to use the simpson and shanon index from \texttt{vegan} package
\end{itemize}

\begin{enumerate}
\def\labelenumi{\arabic{enumi}.}
\setcounter{enumi}{3}
\tightlist
\item
  Individual Attributes
\end{enumerate}

\begin{itemize}
\tightlist
\item
  With individual fixed effects, we partial out any time-invariant
\item
  From the e-bird data:

  \begin{itemize}
  \tightlist
  \item
    How frequently you went in the past year
  \item
    Average distance traveled in the past year
  \item
    Is there any measure of learning over time? Value of recreation is
    not static

    \begin{itemize}
    \tightlist
    \item
      Experience index
    \item
      Cumulative experience (simple counter of trips up unti that point)
    \item
      Could weight this by how much you walk, and taxonomy of bird level
      data (raw text file of trip-species level)
    \item
      Could collapse species count to trip level.
    \end{itemize}
  \end{itemize}
\end{itemize}

\subsection{Project Management}\label{project-management}

\begin{itemize}
\tightlist
\item
  Meetings need to be in person.
\item
  Jovin update meeting calendar
\item
  Start assigning tasks instead of asking for volunteers
\item
  Need to come up with a specialization of tasks.
\item
  Coding and Data: Matt (lead), Jovin, Ryan (assist)
\item
  Conceptualize: Ryan (lead), and Raahil

  \begin{itemize}
  \tightlist
  \item
    How our model departs from workhouse.
  \item
    Literature review for our novelty
  \end{itemize}
\item
  Some tasks:

  \begin{itemize}
  \tightlist
  \item
    Google maps driving time
  \item
    inaturalist species richness mapping
  \end{itemize}
\end{itemize}

\end{tcolorbox}

\begin{tcolorbox}[enhanced jigsaw, bottomrule=.15mm, coltitle=black, opacityback=0, opacitybacktitle=0.6, titlerule=0mm, leftrule=.75mm, arc=.35mm, toptitle=1mm, title=\textcolor{quarto-callout-note-color}{\faInfo}\hspace{0.5em}{5/15/2025}, bottomtitle=1mm, colbacktitle=quarto-callout-note-color!10!white, breakable, colback=white, colframe=quarto-callout-note-color-frame, rightrule=.15mm, toprule=.15mm, left=2mm]

\begin{quote}
Ryan, Matt
\end{quote}

\subsection{Coding Notes}\label{coding-notes}

\begin{itemize}
\tightlist
\item
  Ryan is taking lead on estimating travel cost.
\item
  We need a seperate hotspots shapefile for running zonal stats
\item
  We need unique hotspot site IDs and hotspot cluster IDs
\item
  Explor alternatives to making hotspot clusters
\item
  Name for the master data
\item
  Need to add attributes to hotspot clusters (script 4)
\item
  Mapping of hotspot cluster ID to hotspot IDs
\end{itemize}

\end{tcolorbox}

\begin{tcolorbox}[enhanced jigsaw, bottomrule=.15mm, coltitle=black, opacityback=0, opacitybacktitle=0.6, titlerule=0mm, leftrule=.75mm, arc=.35mm, toptitle=1mm, title=\textcolor{quarto-callout-note-color}{\faInfo}\hspace{0.5em}{5/2/2025}, bottomtitle=1mm, colbacktitle=quarto-callout-note-color!10!white, breakable, colback=white, colframe=quarto-callout-note-color-frame, rightrule=.15mm, toprule=.15mm, left=2mm]

\begin{quote}
Raahil, Matt
\end{quote}

\subsection{Agenda}\label{agenda-4}

\begin{enumerate}
\def\labelenumi{\arabic{enumi}.}
\tightlist
\item
  Site attributes data progress

  \begin{itemize}
  \tightlist
  \item
    In progress
  \end{itemize}
\item
  Individual attributes data progress
\item
  Coding sample selection progress

  \begin{itemize}
  \tightlist
  \item
    Done, flexible to time period and years (e.g.~quarterly for at least
    8 years)
  \end{itemize}
\end{enumerate}

\subsection{Site Attributes}\label{site-attributes}

\begin{itemize}
\tightlist
\item
  Still gathering geospatial data

  \begin{itemize}
  \tightlist
  \item
    \href{https://lpdaac.usgs.gov/products/mod44bv061/}{MODIS VCF forest
    cover}
  \item
    \href{https://cds.climate.copernicus.eu/datasets/reanalysis-era5-land?tab=download}{Copernicus
    ERA5-Land}

    \begin{itemize}
    \tightlist
    \item
      \href{https://bluegreen-labs.github.io/ecmwfr/}{ecmwfr}, R API
      interface
    \end{itemize}
  \end{itemize}
\item
  Run zonal statistics on buffer around hotspots
\item
  If data becomes to large to easily share, we can start using Globus
  and MSI's tier 2 data storage
\end{itemize}

\subsection{Next tasks}\label{next-tasks}

\begin{enumerate}
\def\labelenumi{\arabic{enumi}.}
\tightlist
\item
  Merge site and individual attributes to make master dataset
\item
  Calculate travel cost

  \begin{itemize}
  \tightlist
  \item
    Start with linear distance for now
  \item
    Calculate cost

    \begin{itemize}
    \tightlist
    \item
      Refer to Lloyd-Smith and Cameron/Kolstoe
    \item
      Travel cost = 2 (round trip) x distance x 1/3 x wage (from GDP)
    \end{itemize}
  \end{itemize}
\item
  Species Richness

  \begin{itemize}
  \tightlist
  \item
    Decide on method:

    \begin{itemize}
    \tightlist
    \item
      Weitzman index
    \item
      iNaturalist
    \item
      Simpson index
    \item
      Shannon index
    \end{itemize}
  \end{itemize}
\item
  Looking ahead

  \begin{itemize}
  \tightlist
  \item
    Look at code for estimating mixed logit with big data

    \begin{itemize}
    \tightlist
    \item
      \href{https://github.com/joemolloy/fast-mixed-mnl}{mixl}, C++
      based R package
    \item
      \href{https://xlogit.readthedocs.io/en/latest/index.html}{xlogit},
      GPU-Accelerated Python Package
    \end{itemize}
  \item
    How to properly weight the WTP for aggregation
  \item
    Research how to include fixed effects in utility functions
    (alternative specific constant)

    \begin{itemize}
    \tightlist
    \item
      User fixed effects
    \item
      hotspot fixed effects
    \end{itemize}
  \end{itemize}
\end{enumerate}

\end{tcolorbox}

\begin{tcolorbox}[enhanced jigsaw, bottomrule=.15mm, coltitle=black, opacityback=0, opacitybacktitle=0.6, titlerule=0mm, leftrule=.75mm, arc=.35mm, toptitle=1mm, title=\textcolor{quarto-callout-note-color}{\faInfo}\hspace{0.5em}{4/25/2025}, bottomtitle=1mm, colbacktitle=quarto-callout-note-color!10!white, breakable, colback=white, colframe=quarto-callout-note-color-frame, rightrule=.15mm, toprule=.15mm, left=2mm]

\begin{quote}
Matt, Ryan
\end{quote}

\subsection{Tasks:}\label{tasks}

\begin{itemize}
\tightlist
\item
  Matt: Download data and share
\item
  Ryan: Start writing pseudocode in the new scripts.
\item
  Both: Play with ClimateR
\end{itemize}

\subsection{Site Attributes}\label{site-attributes-1}

\begin{itemize}
\tightlist
\item
  Limitation:

  \begin{itemize}
  \tightlist
  \item
    It takes time to build GIS data. So a lot of the attributes are not
    updated for 2024/25
  \item
    ClimateR
  \end{itemize}
\item
  Division of the attributes to collect

  \begin{itemize}
  \tightlist
  \item
    Protected area (binary or distance) -- fuzzy borders leads to
    spillovers

    \begin{itemize}
    \tightlist
    \item
      Matt
    \item
      WDPA
      (https://www.protectedplanet.net/en/thematic-areas/wdpa?tab=WDPA)
    \end{itemize}
  \item
    Tree cover -- higher willingess to pay for forest

    \begin{itemize}
    \tightlist
    \item
      Matt
    \item
      https://data.fs.usda.gov/geodata/rastergateway/treecanopycover/
    \item
      https://data.globalforestwatch.org/
    \item
      https://glad.umd.edu/dataset/global-2010-tree-cover-30-m
    \item
      Worse comes to worse we use the ESA data
    \item
      Or we could use NDVI data as a proxy.
    \end{itemize}
  \item
    Rainfall (monthly) - seasonality

    \begin{itemize}
    \tightlist
    \item
      Ryan
    \item
      https://ftp.cpc.ncep.noaa.gov/GIS/GRADS\_GIS/GeoTIFF/GLB\_DLY\_PREC/
    \item
      CHRIPS SPI data
    \item
      https://mikejohnson51.github.io/climateR/index.html
    \end{itemize}
  \item
    Temperature (monthly) - seasonality

    \begin{itemize}
    \tightlist
    \item
      Ryan
    \item
      GISS Surface Temperature Analysis (GISTEMP v3)
    \item
      Surface air temperature (no ocean data), 250km smoothing (9 MB)
    \item
      Might need the land mask
    \item
      https://data.giss.nasa.gov/gistemp/
    \end{itemize}
  \item
    Distance to the coast -- species gradient to the coast

    \begin{itemize}
    \tightlist
    \item
      Ryan
    \item
      Dissolve the districts data.
    \item
      Make a polyline and measure distance to the nearest line
    \end{itemize}
  \item
    Species richness from the previous week (expectation) -- and any
    other measures of biodiversity

    \begin{itemize}
    \tightlist
    \item
      Both Matt and Ryan
    \item
      Weitzmen index using ebird data to generate diversity score
    \item
      https://scholar.harvard.edu/files/weitzman/files/on\_diversity.pdf
    \item
      Validate with inaturalist data on birds for india
    \item
      Could test a couple of different indices
    \end{itemize}
  \end{itemize}
\item
  How we are sharing the site attribute data

  \begin{itemize}
  \tightlist
  \item
    Matt will have the GDrive link.
  \end{itemize}
\item
  Updating the config files

  \begin{itemize}
  \tightlist
  \item
    Once we have data download
  \end{itemize}
\item
  Start script to merge site attributes to sites

  \begin{itemize}
  \tightlist
  \item
    What to do about 4.demographics. Seems like that should be site
    attributes.
  \item
    Noteably, some of the data will need to become panel.

    \begin{itemize}
    \tightlist
    \item
      Rainfall
    \item
      Temperature
    \item
      Protected Areas
    \item
      Tree cover
    \item
      species richness (we need to dig some that has a time dimension)
    \end{itemize}
  \item
    Static:

    \begin{itemize}
    \tightlist
    \item
      Distance to coast
    \end{itemize}
  \end{itemize}
\end{itemize}

\end{tcolorbox}

\begin{tcolorbox}[enhanced jigsaw, bottomrule=.15mm, coltitle=black, opacityback=0, opacitybacktitle=0.6, titlerule=0mm, leftrule=.75mm, arc=.35mm, toptitle=1mm, title=\textcolor{quarto-callout-note-color}{\faInfo}\hspace{0.5em}{4/18/2025}, bottomtitle=1mm, colbacktitle=quarto-callout-note-color!10!white, breakable, colback=white, colframe=quarto-callout-note-color-frame, rightrule=.15mm, toprule=.15mm, left=2mm]

\begin{quote}
Raahil, Matt, Jovin, Ryan
\end{quote}

\subsection{Sample Selection Updates}\label{sample-selection-updates}

\begin{itemize}
\tightlist
\item
  different biodiversity levels by incorporating interaction terms in
  locals and tourists. WTP for nature tourism vs WTP for nature
  recreation or biodiverstiy

  \begin{itemize}
  \tightlist
  \item
    If we estimate both, then we can compare both measures. We can also
    place it within the larger literature of many WTP for both subsets
    of ecosystem services and the population using those services.
  \item
    Broadly compare conservation, nature recreation, tourism, or
    biodiversity
  \item
    We need to frame this as a development paper as the lambda in the
    Envirodevonomics paper
  \end{itemize}
\item
  Frequency of observations are consistent over time and potentially
  visits near protected areas.

  \begin{itemize}
  \tightlist
  \item
    Need to capture seasonanlity (time fixed effects)
  \item
    Need to consider some aspect of congestion (can calculate that as a
    site attribute)
  \item
    If you go two months a year you are less likely to be a tourist
  \end{itemize}
\end{itemize}

\subsection{Literature Review Updates}\label{literature-review-updates}

\begin{itemize}
\tightlist
\item
  We can uses species richness as an explanatory variable in the model.
  But could add other biodiversity measures beyond bird diversity.
\item
  Need to think about how to frame this in the developing context
  instead of the environment context
\item
  Very few WTP for environmental quality in development economics space
\item
  Would be great if our WTP measure could be placed within a welfare
  model so we can make counterfactual statements

  \begin{itemize}
  \tightlist
  \item
    Need to have homogenous agents to aggregate for welfare gains
    \textgreater\textgreater{} e-bird users are not representative of
    population
  \item
    Could overcome this by population demographic weights to make e-bird
    users representative
  \item
    But welfare statements are a big improvement over the other e-bird
    papers
  \end{itemize}
\item
  Could explore the WTP measure of space and time. So we go beyond the
  cross-sectional estimates in the literature

  \begin{itemize}
  \tightlist
  \item
    If we have 1st and 2nd moment, then we can assign individuals an
    individual WTP.
  \end{itemize}
\end{itemize}

\subsection{Grant Proposal}\label{grant-proposal}

\begin{itemize}
\tightlist
\item
  Jovin will provide an update.
\end{itemize}

\subsection{Initial Results by Raahil}\label{initial-results-by-raahil}

\begin{itemize}
\tightlist
\item
  We are getting coverage of the country
\item
  Some overlap of population density and users
  \textgreater\textgreater{} likely representative
\item
  Most hotspots and users are in the south (consistent with e-birder
  users)
\item
  Hotspots data could be weighted by visits
\item
  Should plot protected areas overlapping hotspots
\item
  Demographic t-test shows e-bird users are richer on average. But we
  can use this to back out our inverse weight for the welfare
  calculations

  \begin{itemize}
  \tightlist
  \item
    E-bird users are selecting into this where there are richer places
    (so can afford this hobby)
  \end{itemize}
\end{itemize}

\subsection{Tasks}\label{tasks-1}

\begin{itemize}
\tightlist
\item
  How do we distinguish ourselves from the Pacific Northwest paper
  Kolestet 2017
\item
  First pass of the sample: users who use this at least 2 months out of
  the year
\item
  Make two new scripts:

  \begin{itemize}
  \tightlist
  \item
    Merging in site attributes (Can merge in satilette data -- radius
    around the centriod and take the mean -- and district data about the
    site attribtues -- merge on district ID)

    \begin{itemize}
    \tightlist
    \item
      Protected area (binary or distance) -- fuzzy borders leads to
      spillovers
    \item
      Tree cover -- higher willingess to pay for forest
    \item
      Rainfall (monthly) - seasonality
    \item
      Temperature (monthly) - seasonality
    \item
      Distance to the coast -- species gradient to the coast
    \item
      Species richness from the previous week (expectation) -- and any
      other measures of biodiversity
    \end{itemize}
  \item
    Estimating the RUM via the mixed multinominal logit
  \end{itemize}
\item
  Email tag for grant proposals
\item
  Make a script into a single PDF
\end{itemize}

\end{tcolorbox}

\begin{tcolorbox}[enhanced jigsaw, bottomrule=.15mm, coltitle=black, opacityback=0, opacitybacktitle=0.6, titlerule=0mm, leftrule=.75mm, arc=.35mm, toptitle=1mm, title=\textcolor{quarto-callout-note-color}{\faInfo}\hspace{0.5em}{3/31/2025}, bottomtitle=1mm, colbacktitle=quarto-callout-note-color!10!white, breakable, colback=white, colframe=quarto-callout-note-color-frame, rightrule=.15mm, toprule=.15mm, left=2mm]

\begin{quote}
Raahil, Matt, Jovin
\end{quote}

\subsection{Sample Selection
Discussion}\label{sample-selection-discussion}

\begin{itemize}
\tightlist
\item
  We want to ensure that the sample of ebird observers is representative

  \begin{itemize}
  \tightlist
  \item
    Remove tourists
  \item
    Remove infrequent users
  \end{itemize}
\item
  Potential strategies

  \begin{itemize}
  \tightlist
  \item
    Limit to users that log at least a certain number of times per month
  \item
    Exclude users that post outside of protected areas
  \item
    Exclude users that include tourism tags in text
    variable(e.g.~``ecolodge'')
  \end{itemize}
\end{itemize}

\subsection{Tasks}\label{tasks-2}

\begin{itemize}
\tightlist
\item
  Matt/Jovin: descriptive analysis on data to determine distributions of
  frequent users

  \begin{itemize}
  \tightlist
  \item
    Complete by 4/18
  \end{itemize}
\end{itemize}

\end{tcolorbox}

\begin{tcolorbox}[enhanced jigsaw, bottomrule=.15mm, coltitle=black, opacityback=0, opacitybacktitle=0.6, titlerule=0mm, leftrule=.75mm, arc=.35mm, toptitle=1mm, title=\textcolor{quarto-callout-note-color}{\faInfo}\hspace{0.5em}{3/21/2025}, bottomtitle=1mm, colbacktitle=quarto-callout-note-color!10!white, breakable, colback=white, colframe=quarto-callout-note-color-frame, rightrule=.15mm, toprule=.15mm, left=2mm]

\begin{quote}
Raahil, Matt, Ryan, Jovin
\end{quote}

\subsection{Action Items:}\label{action-items}

\begin{itemize}
\tightlist
\item
  Everyone confirms that we can run the first 3 basic scripts
\item
  Starting the annotated bibliography with 10 most important papers:

  \begin{itemize}
  \tightlist
  \item
    WTP for envrionmental quality
  \item
    WTP in developing countries using travel cost
  \item
    Review of mixed logit methodology
  \end{itemize}
\item
  Next week we need to talk about sample selection in depth next week
\end{itemize}

\section{Division of Tasks:}\label{division-of-tasks}

\begin{itemize}
\item
  Review mixed logit method
\item
  (Matt) Review of WTP in developing countries using travel cost methods
\item
  (Ryan) Review of WTP methods for environmental quality
\item
  (Ryan) Method to estimate welfare effects (some exogenous shock)
\item
  (Raahil + Jovin + Matt) Determining the sample population criteria --
  important early step
\item
  (Raahil) Gathering data for population demographics (SHRUG, DHS, etc.)
\item
  Ways to annoate bibliography:

  \begin{itemize}
  \tightlist
  \item
    paper title + journal, research question, research design
    (e.g.~stated preference), data/sample size, result (i.e.~WTP
    estimate)
  \end{itemize}
\item
  We need to think about how we contribute.
\item
  We need to think about sample selection.

  \begin{itemize}
  \tightlist
  \item
    Criteria based on usage, overlap with protected areas, etc.
  \item
    How do we remove tourists?
  \item
    Who is the average person? Estimates will be sensitive
  \end{itemize}
\item
  Need to gridded map of ebird observations
\end{itemize}

\section{Next scripts}\label{next-scripts}

\begin{enumerate}
\def\labelenumi{\arabic{enumi}.}
\setcounter{enumi}{3}
\tightlist
\item
  Merge attributes of the counterfactual sites. Need to collect site
  level data and/or individual level data

  \begin{itemize}
  \tightlist
  \item
    Site attributes: biodiversity, accessibility, ruggedness,
    congestion, temperature, rainfall, regional developmemnt
  \item
    For first run, just do biodiversity.
  \item
    Biodiversity comes from either diversity of birds or other
    attributes of ecosystem diversity
  \item
    Expected biodiversity is why you go. So we can use historical
    measures of biodiversity
  \end{itemize}
\item
  Run the mixed logit to estimate the WTP values
\item
  Welfare analysis of WTP values
\end{enumerate}

\end{tcolorbox}

\begin{tcolorbox}[enhanced jigsaw, bottomrule=.15mm, coltitle=black, opacityback=0, opacitybacktitle=0.6, titlerule=0mm, leftrule=.75mm, arc=.35mm, toptitle=1mm, title=\textcolor{quarto-callout-note-color}{\faInfo}\hspace{0.5em}{02/28/25}, bottomtitle=1mm, colbacktitle=quarto-callout-note-color!10!white, breakable, colback=white, colframe=quarto-callout-note-color-frame, rightrule=.15mm, toprule=.15mm, left=2mm]

\begin{quote}
Raahil, Matt, Ryan
\end{quote}

\subsection{Action Items:}\label{action-items-1}

\begin{itemize}
\tightlist
\item
  Ryan: Get the correct e-bird file from Matt (DONE)
\item
  Ryan: Merge code files into scripts folder. (DONE)
\item
  Ryan: Initate a document for literature review with template example
  (DONE)
\item
  We need to specalize and define tasks. Volunteer over email. Tasks:
  Lit review, sample selection, characterize the sample, theory person
  (DONE)
\item
  And set up a recurring meeting time. Need a poll. Try for next two
  weeks from now. (DONE)
\item
  Jovin: Provide your github username so Ryan can make you a
  collaborator on the github repo. (DONE)
\end{itemize}

\subsection{Funding:}\label{funding}

\begin{itemize}
\tightlist
\item
  No longer pursue IonE funding
\end{itemize}

\subsection{Coding:}\label{coding}

\begin{itemize}
\tightlist
\item
  Matt was able to replicate everything

  \begin{itemize}
  \tightlist
  \item
    Created a YAML to process file names
  \item
    No longer needs new file paths
  \end{itemize}
\item
  Ryan stuck on loading data
\item
  Need to all use the same exact file download for e-bird
\item
  script 1: processes e-bird data, determines user homes, and calculates
  home for those who don't report it.
\item
  script 2: Creates the hotspot observations. Measures distance from
  home to hotspot.
\item
  script 3: Most important. 12,000 hotspots so too many counterfactuals.
  This script reduces the counterfactual set by clustering hotspots
  nearby (likley parks) and only consider hotspots within a distance of
  home.
\item
  The final data set is a panel of person (user) ID by trip ID appended
  to all counterfactual trips for this user.
\item
  When we actually estimate this, we will need to plan for a no-trip
  counterfactual.
\item
  One of the main things we need to do is make the computation easier
  for us. Otherwise it becomes too computationally intensive.
\item
  Keep all the scripts in scripts folder.
\end{itemize}

\subsection{Data:}\label{data}

\begin{itemize}
\tightlist
\item
  We will need to bring in site attribute data
\item
  So data on landscape measurements.
\item
  Matt says that not all intermediate files are being save
\item
  Specifically, the master file is not saving
\item
  Need to identify our sample population

  \begin{itemize}
  \tightlist
  \item
    Need a careful (defensiable) data-driven criteria for sample
  \item
    Needs to be regular, local users (not temporary or tourists)
  \item
    Determine this based on usage frequency and mobility patterns
  \item
    Consider overlaying protected areas polygons on top of the
    observations/hotspots
  \item
    Buffers may create a censoring problem. So we would just test this
    against larger and larger buffers
  \end{itemize}
\item
  Need to find ways to characteristize the users (demographics) without
  having any information about them.

  \begin{itemize}
  \tightlist
  \item
    Important for welfare calculations.
  \item
    Try to match with DHS data
  \item
    Or overlap home location with SHRUG data set
  \item
    Econometrically, take attribtues about where they live against the
    average for the country, how do high and low WTP birders fall within
    the distribution
  \item
    Potentially use this as weights for welfare calculations.
  \end{itemize}
\end{itemize}

\subsection{Methods:}\label{methods}

\begin{itemize}
\tightlist
\item
  If we get a random shock to bird population, then we can think about
  the causal effect of increasing biodiversity
\end{itemize}

\subsection{Paper/Writing:}\label{paperwriting}

\begin{itemize}
\tightlist
\item
  For the ML cluster, will want to read up later to find out if there is
  a better way or more common practice in economics.
\item
  Create a lit review. Three groups:

  \begin{enumerate}
  \def\labelenumi{\arabic{enumi}.}
  \tightlist
  \item
    estimates of WTP in developing countries generally using the
    discrete choice travel cost method (Ted Miguel water quality paper).
  \item
    WTP for environmental quality papers and what design they are using
    to estimate (e.g., contingent valuation, stated preferences --
    something from ecology)
  \item
    Review of any updated versions of the mixed logit model -- new
    adapations of the model from class
  \end{enumerate}
\item
  For this: question, data used, design, and finding (dollar estimates
  would be great).
\item
  Create a document in the docs folder
\end{itemize}

\end{tcolorbox}




\end{document}
